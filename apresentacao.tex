\documentclass{beamer}
\usetheme{Berlin}
\usecolortheme{crane}
\usepackage[brazilian]{babel}
\usepackage[utf8]{inputenc}
\usepackage[T1]{fontenc}
\usepackage{graphicx}
\title{Invasão da Equifax}
\author[Athos Castro Moreno]{Athos Castro Moreno}

\institute{Universidade Tecnológica Federal do Paraná}

\titlegraphic{\includegraphics[scale=0.15]{utfpr-logo.png}}

\date{\today}

\begin{document}
	\begin{frame}
		\titlepage
	\end{frame}
	\begin{frame}
		\frametitle{Sumário}
		\tableofcontents
	\end{frame}
	\section{Introdução}
	\begin{frame}
		\frametitle{Introdução}
		\begin{itemize}
			\item Equifax: bureau de crédito existente desde 1899;
			\item Três principais do mercado: Equifax, Experian, TransUnion;
			\item Equifax fundiu com a Boa Vista em 2011 no Brasil;
			\item Experian atua com a marca Serasa Experian no Brasil. 
		\end{itemize}
	\end{frame}
	\section{Descoberta da invasão}
	\begin{frame}
		\frametitle{Descoberta da invasão}
		\begin{itemize}
			\item Anuncio feito dia sete de setembro de 2017;
			\item 143 milhões de pessoas tiveram suas informações comprometidas;
			\item O acesso não autorizado aconteceu entre maio e julho de 2017;
			\item No momento do anúncio, era conhecido que a invasão aconteceu através de uma vulnerabilidade no website da empresa.
		\end{itemize}
	\end{frame}
	\section{Vulnerabilidade}
	\begin{frame}
		\frametitle{Vulnerabilidade}
		\begin{itemize}
			\item Vulnerabilidade no \textit{framework} Apache Struts 2;
			\item \textit{Framework} para projetos web em Java;
			\item CVE-2017-5638;
			\item Permitia a um invasor enviar uma requisição modificada para um servidor vulneravel;
			\item O invasor enviava código malicioso no \textit{header} Content-Type;
			\item O código malicioso era executado na maquina vulneravel.
		\end{itemize}
	\end{frame}
	\begin{frame}
		\frametitle{Vulnerabilidade}
		\begin{itemize}
			\item A vulnerabilidade no \textit{framework} Struts 2 foi divulgada em março de 2017;
			\item Desde a divulgação já tinham potenciais explorações em servidores em produção, sendo considerada uma falha critica;
			\item Os sistemas da Equifax ficaram sem a correção até julho de 2017.
			\item Tal fato foi comentado na imprensa como uma negligencia grosseira da Equifax;
			\item O acontecimento foi comentado pela equipe da Apache Struts 2, que enfatizou a importancia de sistemas atualizados.
		\end{itemize}
	\end{frame}
	\section{Falhas após a invasão}
	\begin{frame}
		\frametitle{Falhas após a invasão}
		\begin{itemize}
			\item No dia seguinte, a Equifax anunciou que disponibilizou o serviço de proteção a crédito \textit{TrustedID Premier} gratuitamente para quem deseja verificar se foi impactado;
			\item Os termos de serviço do \textit{TrustedID Premier} incluiam uma clausula que forçava o consumidor que se cadastra-se no serviço a não participar de ações conjuntas contra a Equifax;
			\item Diversos processos relacionados a invasão apareceram logo após o anúncio;
		\end{itemize}
	\end{frame}
	\begin{frame}
		\frametitle{Falhas após a invasão}
		\begin{itemize}
			\item A Equifax criou um website para facilitar o cadastro no \textit{TrustedID Premier};
			\item Ao invés de utilizarem o dominio atual da empresa, \url{https://www.equifax.com}, registraram um dominio novo;
			\item O dominio novo era relativamente longo e facil de ser confundido;
			\item Logo após o anuncio, um profissional de segurança registrou um dominio semelhante;
			\item O dominio falso mostrava uma página semelhante a da Equifax, mas continha um texto sarcástico com relação ao uso do dominio. 
		\end{itemize}
	\end{frame}
	\begin{frame}
		\frametitle{Falhas após a invasão}
			\begin{itemize}
				\item Dominio da Equifax: \url{https://www.equifaxsecurity2017.com}
				\item Dominio com website falso: \url{https://www.securityequifax2017.com}
			\end{itemize}
	\end{frame}
	\begin{frame}
		\frametitle{Falhas após a invasão}
		\begin{itemize}
			\item O ex-CEO da Equifax, Richard F. Smith, afirmou que a invasão foi causada por falha de um único funcionário;
			\item Um outro website da Equifax para utilizar o serviço \textit{TrustedID Premier} possuia uma vulnerabilidade XSS.
		\end{itemize}
	\end{frame}
	\section{Conclusão}
	\begin{frame}
		\frametitle{Conclusão}
		\begin{itemize}
			\item Não são conhecidos os invasores da Equifax;
			\item Suponha-se que são russos ou chineses, com possibilidade de apoio estatal;
			\item Além dos processos, existem investigações no Senado dos Estados Unidos;
			\item Nenhuma sanção foi aplicada a empresa;
			\item Começou-se a discutir a validade do uso do \textit{Social Security Number};
		\end{itemize}
	\end{frame}
	\begin{frame}
		\frametitle{Conclusão}
		\begin{itemize}
			\item Importância da segurança da informação;
			\item Importância de planos de contigencia;
			\item Transparência com o público;
			\item Atualização de monitoramento de \textit{softwares} e \textit{frameworks} utilizados.
		\end{itemize}
	\end{frame}
	\begin{frame}
		\titlepage
	\end{frame}
\end{document}
