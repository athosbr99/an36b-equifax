\documentclass[conference]{IEEEtran}
\usepackage{cite}
\usepackage{amsmath,amssymb,amsfonts}
\usepackage{algorithmic}
\usepackage{graphicx}
\usepackage{textcomp}
\usepackage[utf8]{inputenc}
\def\BibTeX{{\rm B\kern-.05em{\sc i\kern-.025em b}\kern-.08em
    T\kern-.1667em\lower.7ex\hbox{E}\kern-.125emX}}
\begin{document}

\title{Invasão da Equifax}

\author{\IEEEauthorblockN{Athos Castro Moreno}
    \IEEEauthorblockA{\textit{Departamento de Computação} \\
    \textit{Universidade Tecnológica Federal do Paraná}\\
    Cornélio Procópio, Paraná, Brasil \\
    athos@alunos.utfpr.edu.br}
}

\maketitle

\begin{abstract}

\end{abstract}

\begin{IEEEkeywords}

\end{IEEEkeywords}

\section{Introdução}

\section{Descoberta da invasão} 

No dia sete de setembro de 2017, a Equifax anunciou que 143 milhões de pessoas residentes nos Estados Unidos da América 
tiveram suas informações pessoas comprometidas por invasores ainda não identificados \cite{Carman2017}.

\bibliographystyle{IEEEtran}
\bibliography{bibliografia}

\end{document}
