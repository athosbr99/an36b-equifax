\documentclass[conference]{IEEEtran}
\usepackage{cite}
\usepackage{amsmath,amssymb,amsfonts}
\usepackage{algorithmic}
\usepackage{graphicx}
\usepackage{textcomp}
\usepackage{hyperref}
\usepackage[utf8]{inputenc}
\def\BibTeX{{\rm B\kern-.05em{\sc i\kern-.025em b}\kern-.08em
    T\kern-.1667em\lower.7ex\hbox{E}\kern-.125emX}}
\begin{document}

\title{Invasão da Equifax}

\author{\IEEEauthorblockN{Athos Castro Moreno}
    \IEEEauthorblockA{\textit{Departamento de Computação} \\
    \textit{Universidade Tecnológica Federal do Paraná}\\
    Cornélio Procópio, Paraná, Brasil \\
    athos@alunos.utfpr.edu.br}
}

\maketitle

\begin{abstract}
No final de 2017, foi descoberto uma invasão nos sistemas da Equifax, uma bureau de crédito dos Estados Unidos e uma das maiores do ramo. A invasão impactou diretamente milhares de residentes nos
Estados Unidos, além de outros países. Pouco se conhece sobre os invasões e o que fizeram com os dados roubados. A gravidade da invasão, a maneira como os invasões obteram acesso e os erros da Equifax
após o anuncio da invasão fizeram deste acontecimento um dos maiores da história da segurança da informação.
\end{abstract}

\begin{IEEEkeywords}
equifax, invasão, segurança, informações, privacidade
\end{IEEEkeywords}

\section{Introdução}

\section{A empresa}
A Equifax Inc. é uma bureau de crédito criada em 1899 com o nome de \textit{Retail Credit Company} que atua mundialmente e possui sede em Atlanta, Georgia, nos Estados Unidos. 
É uma das três maiores agencias de crédito do mundo, junto com a Experian -- que atua no Brasil com a marca Serasa Experian,
ou somente Serasa -- e a TransUnion \cite{Roos2008}.

Além de oferecer informações relacionadas a crédito de consumidores e dados demográficos para empresas e bancos, a Equifax
vende serviços de monitoramento de crédito e anti fraude a pessoas físicas. Nos Estados Unidos, devido regulação, fornecem um
relatório anual gratuito de crédito aos cidadãos \cite{Roos2008} \cite{Garkinkel1995}. 

No Brasil, a Equifax iniciou suas operações em 1998. Em 2011, fundiu suas operações para a Boa Vista Serviços, que administra o Serviço de Proteção ao Crédito, também conhecido
como SPC \cite{BoaVista2018} \cite{Sandrini2011}. 

\section{Descoberta da invasão} 
No dia sete de setembro de 2017, a Equifax anunciou que 143 milhões de pessoas residentes nos Estados Unidos da América tiveram suas informações pessoas comprometidas por invasores 
não identificados. De acordo com a investigação interna da empresa, os acessos dos invasores ocorreram entre o meio de maio até julho de 2017, sendo a invasão identificada e remediada 
em 29 de julho de 2017. No momento do anuncio, era conhecido que a invasão aconteceu através de uma vulnerabilidade em um website da empresa. \cite{Carman2017} \cite{Equifax2017} \cite{Bernard2017} \cite{Moore2017}.

No começo de 2018, a Equifax anunciou que a invasão afetou mais pessoas do que o número divulgado anteriormente, chegando a aproximadamente 148 milhões de pessoas afetadas \cite{Whittaker2018} \cite{Borak2018} \cite{Clements2018}.

No dia 18 de março de 2018, o ex-executivo da Equifax Jun Ying foi processado pela \textit{Securitites and Exchange Commission} dos Estados Unidos por \textit{insider trading}, que é o uso de informações confidenciais
para ganho próprio no mercado de ações. Dez dias antes da divulgação da invasão, Ying vendeu todas as suas ações da Equifax, evitando uma perda de mais de 117 mil dólares. 

\section{Vulnerabilidade}
O sistema da Equifax foi invadido através de uma vulnerabilidade no \textit{framework} Apache Struts 2, utilizado para o desenvolvimento de aplicações web em larga escala na linguagem Java. A vulnerabilidade explorada
foi a CVE-2017-5638, que permite um invasor enviar uma requisição para o envio de um arquivo modificada para um servidor vulnerável. O invasor pode enviar então, código malicioso no \textit{header} Content-Type para
executar comandos no servidor infectado \cite{Luszcz2018} \cite{Sahu2017}.

A falha no código do Apache Struts 2 foi divulgada em março de 2017, com potenciais explorações em sistemas em produção, sendo considerada uma falha critica. Desta maneira, os 
sistemas da Equifax ficaram sem a atualização com a correção da CVE-2017-5638 até julho, o que permitiu o acesso aos invasores. Tal fato foi comentado e criticado na imprensa, 
sendo uma negligencia grosseira de segurança por parte da Equifax. O acontecimento também foi comentado pela equipe da Apache Struts, que lamentou o caso e afirmou que 
atualizações de segurança não devem ser negligenciadas e é responsabilidade dos administradores manter seus sistemas atualizados e seguros \cite{Newman2017} \cite{Goodin2017} 
\cite{Dignan2017} \cite{Struts20171} \cite{Struts20172}.

\section{Falhas após a invasão}
No dia seguinte ao anuncio da invasão, a empresa anunciou que disponibilizou seu serviço de proteção a crédito \textit{TrustedID Premier} gratuitamente quem desejar verificar se foi impactado pela invasão. 
Porém, momentos após o serviço ser liberado, diversos veículos de noticias anunciaram que os termos de serviço do \textit{TrustedID Premier} incluíam uma clausula que forçava o consumidor que se cadastra-se
no serviço a não participar de ações conjuntas contra a Equifax na justiça. Logo após a invasão, já existiam processos relacionados a brecha de segurança \cite{Mosendz2017} \cite{Robertson2017} \cite{Grant2017}.

A Equifax criou um website para facilitar que os usuários se cadastrem no \textit{TrustedID Premier}. A URL do website é \url{https://www.equifaxsecurity2017.com}. Diversos usuários, jornalistas e profissionais 
questionaram a escolha do domínio registrado, que é fácil de ser confundido, é semelhante a domínios utilizados para \textit{phishing} e não tem relação direta com o domínio da empresa, \url{https://www.equifax.com}.
Um funcionário da Equifax no Twitter chegou a confundir o endereço e enviar para clientes a URL \url{https://www.securityequifax2017.com}, que no momento estava com uma página semelhante a Equifax, porém com um texto
informando que a página era falsa. O domínio foi registrado por um profissional de segurança momentos depois da Equifax anunciar a brecha e o serviço \cite{Mak2017} \cite{Deahl2017} \cite{Burns2017}. 

Um outro website da Equifax para utilizar o serviço de monitoramento de crédito possuía, pelo menos, uma vulnerabilidade de \textit{cross-site scripting}, também conhecido como ataque XSS. Através deste ataque, um
usuário malicioso poderia modificar a URL da Equifax de maneira que os dados submetidos na página fossem enviados ao usuário malicioso \cite{Whittaker2017}.

\bibliographystyle{IEEEtran}
\bibliography{bibliografia}

\end{document}
